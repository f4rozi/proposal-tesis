\section{Metodologi Penelitian}
Metodologi yang digunakan dalam melakukan penelitian ini adalah sebagai berikut:
\begin{enumerate}
	\item Studi literatur\\
	Tahapan ini dilakukan dengan cara mengumpulkan dan mempelajari literatur-literatur yang berkaitan dengan \emph{Semantic Web}, \emph{protege},\emph{OWL-DL}, \emph{Sparql-DL}, seperti struktur kalimat bahasa alami untuk bahasa Indonesia, metode pengembangan ontologi serta literatur tentang deduksi pengetahuan dengan menggunakan \emph{reasoner}.
	\item Analisis dan perancangan sistem\\
	Tahapan analisis dan perancangan sistem dilakukan secara bertahap, dimulai dari perancangan Sistem kemudian	Setelah proses perancangan ontologi selesai, dilanjutkan dengan proses perancangan sistem utama yaitu meliputi perancangan ontologi dan \emph{query engine}. Pemodelan rancangan menggunakan UML. Proses akhir dari tahapan perancangan adalah perancangan antar muka sistem.
	\item Implementasi hasil perancangan\\
	Tahapan implementasi dilakukan sesuai dengan urutan proses perancangan, yaitu mulai dari realisasi pengembangan ontologi. Realisasi pengembangan ontologi menggunakan tool Protege versi 5.0 beta. 21 Versi ini dipilih karena sudah mendukung penuh pengembangan ontologi dengan bahasa OWL 2.

	Realisasi sistem menggunakan bahasa Java, JSP dan JavaScript, OWL API versi 4 dan untuk tool reasoning menggunakan \emph{reasoner Hermit}. Sedangkan untuk server menggunakan Apache Tomcat versi 8.
	\item Pengujian\\
	Tahapan pengujian dilakukan untuk membuktikan bahwa sistem yang dikembangkan telah bekerja sesuai dengan yang diinginkan. Proses pengujian terdiri dari \emph{unit testing} dan \emph{black box testing} dimana sistem diberikan pertanyaan untuk mengamati keluaran masing-masing fungsi serta apakah jawaban yang diberikan telah sesuai dengan yang dikehendaki atau tidak.

	Beberapa pertanyaan yang dibuat melibatkan data dari beberapa ontologi yang dibangun, hal ini untuk menguji apakah \emph{plugin} yang dikembangkan ini sudah benar-benar dapat memperoses ekspresi yang melibatkan idividual tanpa developer harus mengetahui struktur dan relasi yang dibuat.
	\item Penarikan kesimpulan\\
	Setelah proses pengujian selesai, tahap selanjutnya adalah merangkum semua hasil pengujian untuk dijadikan sebuah kesimpulan mengenai hasil pengembangan sistem termasuk apabila terdapat saran dan penyempurnaan untuk penelitian selanjutnya.
\end{enumerate}