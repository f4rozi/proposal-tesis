\section{Latar Belakang}

Informasi yang terdapat di dalam internet telah banyak memberikan kita kemudahan dalam berbagai hal, mulai dari bidang kesehatan, politik, kehidupan sosial, sejarah, dan berbagai bidang disiplin ilmu lainnya, semua informasi tersebut disajikan dalam bentuk ribuan website atau lebih yang yang membentuk sebuah metadata yang begitu besar, untuk memudahkan pencarian informasi yang kita inginkan mesin pencari yang biasa kita gunakan adalah google.

Mesin pencari biasanya memberikan kita begitu banyak alamat website mengenai informasi yang kita cari tidak secara spesifik langsung mengarahkan kita hanya pada satu halaman website yang kita butuhkan, maka untuk mengatasi masalah tersebut maka muncullah bidang ilmu mengenai \emph{semantic web} untuk menangani masalah tersebut. \emph{Semantic web} membahas bagaimana cara membangun basis pengetahuan atau ontologi agar komputer atau mesin pencari tidak hanya terbatas pada menyajikan informasi saja akan tetapi mengerti dan memahami secara kontekstual informasi yang kita cari sehingga komunikasi yang terjadi antara komputer yang satu dengan yang lainnya dapat terjadi secara \emph{autonomous}. 

Salah satu perangkat lunak yang digunakan untuk membantu dan memudahkan developer dalam membangun ontologi yang berbasis \emph{semantic web} adalah protege. Perangkat lunak protege sangat membantu dan memudahkan developer dalam membangun ontologi, protege memungkinkan ontologi developer untuk mengembangkan ontologi dengan ekspresi deskripsi logic yang sangat kompleks seperti, kelas axiom, \emph{object property axiom}, ekuivalen kelas, ekuivalen property dan lain-lain. Pada protege untuk memeriksa hasil dari sebuah \emph{DL-Sintax}, ontologi developer dapat menggunakan \emph{plugin DL-Query} yang ada pada protege, namun permasalahannya adalah \emph{plugin} pada \emph{DL-Query} hanya mampu memperoses \emph{rule-rule} yang melibatkan ekspresi kelas dengan \emph{object property} saja dan \emph{DL-Query}  tidak mampu memperoses ekspresi yang melibatkan \emph{individual}.

Berdasarkan permasalahan tersebut maka untuk mengatasinya ada satu cara yang bisa digunakan untuk memperoses ekpresi yang melibatkan individual pada protege dengan memanfaatkan \emph{query sparql}, namun \emph{query sparql} juga mempunyai kekurangan yaitu jika ingin memperoses ekspresi yang melibatkan individual maka ontologi developer harus mengetahui nama property yang digunakan. Sehingga dalam membentuk \emph{query sparql} , developer aplikasi harus benar-benar mengetahui struktur dan relasi dalam ontologi yang akan di query, untuk itu diperlukan metode baru yang bisa melengkapi kekurangan dari \emph{DL-Query}.

Menurut \citet{evren_parsia} \emph{Sparql-DL} secara signifikan lebih ekpresif dari pada \emph{DL-Query} yang ada, dimana dengan \emph{Sparql-DL} kita dimungkinkan untuk menggabungkan antara \emph{T-Box}, \emph{R-Box} dan \emph{A-Box query}, dimana DL-query hanya mampu melakukan \emph{query} pada salah satu T-Box, R-Box dan A-Box.

Untuk melengkapi kekurangan \emph{DL-Query} dalam protege yang hanya mampu melakukan proses pada salah satu query yaitu \emph{Rbox query}, \emph{T-box query} dan \emph{Abox query}, maka peneliti mengusulkan pengembangan \emph{plugin} baru dengan menggunakan metode \emph{Sparql-DL} untuk memudahkan ontologi developer dalam membangun ontologi.

