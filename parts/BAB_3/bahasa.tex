\section{Bahasa} % (fold)
\label{sec:section_name}
Menurut \citet{chaer}, bahasa merupakan suatu lambang berupa bunyi, bersifat arbiter, digunakan oleh suatu masyarakat tutur untuk bekerja sama, berkomunikasi dan mengidentifikasi diri. \citet{dardjo}, bahasa adalah suatu lambang berupa bunyi, bersifat arbiter, digunakan oleh suatu masyarakat untuk bekerjasama, berkomunikasi,dan mengidentifikasi diri.
sebagai suatu sistem, bahasa memiliki aturan, kaidah atau pola-pola tertentu, baik dalam tata bunyi, tata bentuk kata, maupun tata kalimat. bahasa disusun oleh tiga komponen, yaitu: sintaksis, fonologi, dan semantik. Komponen sintaksis adalah komponen yang menangani ihwal yang berkaitan dengan kata, frasa dan kalimat. Komponen fonologi adalah komponen yang menangani ihwal yang berkaitan dengan bunyi. komponen semantik adalah komponen yang membahas ihwal makna \citep{dardjo}.

\subsection{Kata}
Menurut \citet{chaer} melalui \citet{admodjo} bahwa kata merupakan perwujudan bahasa sehingga bahasa tidak akan ada jika kata tidak ada. Setiap kata mengandung konsep makna dan mempunyai peranan dalam pelaksanaan bahasa. \citet{alwi} mengelompokkan kata ke dalam empat kategori sintaksis utama yaitu: verba, nomina, adjektiva dan adverbia.

\subsubsection{Verba (kata kerja)}
Kata-kata yang termasuk dalam kelompok kata kerja adalah kata-kata yang dapat diikuti oleh frasa \emph{dengan..} \citet{chaer}. Secara umum, verba berfungsi sebagai predikat atau inti predikat di dalam suatu kalimat \citep{alwi}. Dilihat dari perilaku semantisnya, verba memiliki makna inheren yang terkandung di dalam verba itu sendiri. Verba dapat mengandung makna inheren \emph{perbuatan (aksi), proses} atau \emph{keadaan yang bukan sifat atau kuantitas} \citep{alwi}.

\subsubsection{Adjektiva (kata sifat)}
Adjektiva adalah kata yang memberikan keterangan yang lebih khusus tentang sesuatu yang  dinyatakan oleh nomina dalam suatu kalimat \citep{alwi}. 

\subsubsection{Adverbia (kata keterangan)}
Adverbia adalah kata-kata yang digunakan untuk memberi penjelasan pada kalimat atau bagian kalimat lain, yang sifatnya tidak menerangkan keadaan atau sifat \citep{chaer}. Adverbia perlu dibedakan dalam tataran frasa dan dalam tataran klausa. Dalam tataran frasa, adverbia adalah kata yang menjelaskan verba, adjektiva dan adverbia lain. Sedangkan dalam tataran klausa, adverbia mewatasi atau menjelaskan fungsi-fungsi sintaksis \citep{alwi}.

\subsubsection{Nomina (kata beda)}
Kata benda adalah kata-kata yang dapat diikuti oleh frasa \emph{yang} ... atau \emph{yang sangat} ... \citep{chaer}. Dari segi bentuk morfologisnya, nomina dapat dibagi menjadi nomina dasar dan nomina turunan. Nomina dasar adalah nomina yang terdiri atas satu morfem sedangkan nomina turunan adalah nomina yang diturunkan melalui proses afiksasi, perulangan dan pemajmukan. Dari sisi semantisnya, kata benda mengacu pada manusia, binatang, benda dan konsep atau pengertian. \citet{alwi} mengungkapkan ciri-ciri nomina secara sintaksis adalah sebagai berikut:
\begin{enumerate}
	\item Dalam kalimat yang predikatnya berupa verba, nomina cenderung menduduki posisi sebagai subjek, objek atau pelengkap.
	
	\item Nomina tidak dapat diingkarkan dengan kata tidak.
	
	\item Nomina umumnya dapat diikuti dengan adjektiva, baik secara langsung maupun dengan diantarai oleh kata yang.
\end{enumerate}

\subsection{Frasa}
Frasa adalah gabungan dua buah kata atau lebih yang merupakan satu kesatuan. Tujuan dari penggabungan dua kata atau lebih menjadi satu kesatuan adalah untuk menampung konsep makna yang lebih khas atau lebih tertentu yang tidak dapat diwujudkan dengan sebuah kata saja \citep{chaer}. Secara sintaksis, frasa dapat dikelompokkan ke dalam frasa verbal, frasa nominal, frasa pronominal, frasa numeralia dan frasa preposisional. 

\subsection{Kalimat}
menurut \citet{alwi} melalui \citet{admodjo} kalimat adalah rentetan kata yang disusun sesuai kaidah yang berlaku. Tiap kata dalam kalimat mempunyai tiga klasifikasi, yaitu : kategori sintaksis, fungsi sintaksis, peran semantik. klasifikasi tersaji pada gambar 

\begin{figure}[ht]
	\centering
	\begin{lstlisting}[language=Prolog,xleftmargin=0pt]
	Kalimat            = Ibu     Memarahi   adi	

	Kategori Sintaksis = Nomina  Verba      Nomina 
	fungsi Sintaksis   = Subjek  Predikat   objek
	Peran semantik     = Pelaku  Perbuatan  Sasaran
	\end{lstlisting}
\caption{Klasifikasi Kata dalam sebuah kalimat}
\label{fig:klasifikasi_kata_dalam_kalimat}
\end{figure}

Menurut \citet{alwi}, kalimat merupakan konstruksi sintaksi terbesar yang terdiri dari dua kata atau lebih. Penggabungan dua kata atau lebih dalam satu kalimat menuntut adanya keserasian diantara unsur-unsur tersebut, baik dari segi makna maupun dari segi bentuk.

Kalimat dapat dikelompokkan berdasarkan jumlah kalusa yang menyusun kalimat. Klausa merupakan satuan sintaksis yang terdiri dari dua kata atau lebih yang mengandung unsur predikasi. Menurut \citet{alwi}, berdasarkan jumlah klausanya, kalimat dapat dibedakan menjadi:
\begin{enumerate}
	\item kalimat tunggal\\
	Kalimat tunggal adalah kalimat yang terdiri atas satu klausa. sehingga konstituen untuk unsur subjek dan predikat hanya ada satu atau merupakan datu kesatuan. Kalimat tunggal dapat mengandung unsur manasuka seperti keterangan tempat, waktu dan alat.

	\citet{alwi} melalui \citet{suryawan} membagi kalimat tunggal berdasarkan kategori predikatnya ke dalam kalimat  berpredikat verbal, kalimat berpredikat adjektival, kalimat berpredikat nominal (termasuk pronominal), kalimat berpredikat numeral dan kalimat berpredikat preposisional.

	\item Kalimat majemuk\\
	kalimat majemuk disusun oleh lebih dari satu klausa. Klausa-klausa yang terdapat di dalam kalimat majemuk bertingkat dapat dihubungkan dengan dua cara, yaitu: koordinasi dan subordinasi.

	Koordinasi adalah menghubungkan dua klausa atau lebih yang memiliki kedudukan konstituen sama. Kalimat majemuk yang dibentuk dengan menggunakan koordinasi disebut dengan \emph{kalimat majemuk setara}.

	Subordinasi adalah menguhubungkan dua klausa atau lebih yang kedudukan konstituennya tidak sama. Hubungan yang dibangun dengan menggunakan subordinasi dapat bersifat melengkapi (komplementatif) atau bersifat mewatasi atau menerangkan (atributif). Kalimat majemuk yang dibentuk dengan menggunakan subordinasi disebut dengan \emph{kalimat majemuk bertingkat}.
\end{enumerate}

Kalimat dapat dikelompokkan berdasarkan bentuk atau kategori sintaksis dari kalimat. Berdasarkan bentuk atau kategori sintakasisnya, kalimat dapat dibagi ke dalam empat kelompok, yaitu:
\begin{enumerate}
	\item Kalimat berita (kalimat deklaratif)\\
	Kalimat deklaratif umumnya digunakan untuk menyampaikan pernyataan sehingga isinya merupakan berita bagi pendengar atau pembacanya. Kalimat deklaratif tidak memiliki markah khusus, sehingga kalimat berita dapat berupa bentuk apa saja. Dalam bentuk tulisan, kalimat deklaratif diakhiri dengan tanda titik. Dalam bentuk lisan, kalimat berita diakhiri dengan nada turun.

	\item Kalimat perintah (kalimat imperatif)\\
	\citet{dardjo_et_al} mendefiniskan kalimat imperatif sebagai kalimat yang maknanya memberikan perintah untuk melakukan sesuatu. Dalam bentuk tulisan, kalimat imperatif diakhiri dengan menggunakan tanda seru (!). Dalam bentuk lisan, kalimat imperatif ditandai dengan nada yang semakin tinggi.

	\item Kalimat tanya (kalimat interogatif)\\
	\citet{dardjo_et_al} mendefinisikan kalimat interogatif sebagai kalimat yang isinya menanyakan sesuatu atau seseorang. Kalimat interogatif secara formal ditandai dengan kehadiran kata tanya seperti \emph{apa}, \emph{siapa}, \emph{berapa}, \emph{kapan} dan \emph{bagaimana}, dengan atau tanpa partikel penegas \emph{-kah}. Pada bahasa tulis, kalimat interogatif ditandai dengan tanda tanya (?). Pada bahasa lisan, kalimat tanya ditandai dengan suara naik jika kalimat memiliki kata tanya dan suara turun jika kalimat tidak memiliki kat tanya \citep{alwi}.

	Kalimat interogatif umumnya dibentuk dari kalimat deklaratif. Berikut adalah kaidah pembentukan kalimat tanya dari kalimat deklaratif menurut \citet{alwi}:

	\begin{enumerate}
		\item Kalimat tanya dapat dibentuk dari kalimat deklaratif dengan menambahkan pronomina penanya \emph{apa} pada kalimat tersebut. Partikel \emph{-kah} dapat ditambahkan pada pronomina penanya untuk mempertegas pertanyaan tersebut.

		\item Kalimat tanya dapat dibentuk dengan cara mengubah urutan kata dari kalimat deklaratif. Kaidah yang perlu diperhatikan dalam pembentukan kalimat tanya adalah sebagai berikut:
		\begin{enumerate}
			\item Jika dalam kalimat deklaratif terdapat kata seperti \emph{dapat, bisa, harus, sudah} dan \emph{mau}, maka kata tersebut dipindahkan ke awal kalimat dan ditambahkan partikel \emph{-kah}.

			\item Jika predikat dari kalimat berupa nomina atau adjektiva, urutan subjek dan predikatnya dapat dibalik kemudian partikel \emph{-kah} ditambahkan pada frasa yang telah dipindahkan ke awal kalimat.

			\item Jika predikat kalimat berupa verba taktrasitif, ekatransitif atau semitransitif, maka verba beserta objek atau pelengkapnya dapat dipindahkan ke awal kalimat dan kemudian ditambahkan partikel \emph{-kah}.
		\end{enumerate}

		\item Kalimat tanya dapat dibentuk dengan menempatkan kata \emph{bukan/bukankah, (apa/apakah) belum} atau \emph{tidak}.

		\item Kalimat tanya dapat dibentuk dengan mempertahankan urutan kata seperti dalam kalimat deklaratif, tetapi pengucapannya menggunakan intonasi yang naik.

		\item Kalimat tanya dapat dibentuk dengan menggunakan pronomina interogatif seperti \emph{apa, siapa, mengapa, kenapa, kapan, (Ke)berapa, dimana, kemana, dari mana, bagaimana} dan \emph{bilamana}.
	\end{enumerate}

	\item Kalimat seru (kalimat eksklamatif)\\
	Kalimat seru atau kalimat eksklamatif juga dikenal dengan sebutan kalimat interjeksi. Kalimat seru dapat digunakan untuk menyatakan perasaan kagum atau heran \citep{alwi}. \citet{chaer} menyatakan bahwa kalimat seru juga dapat digunakan untuk menyatakan emosi atau perasaan batin yang biasanya terjadi secara tiba-tiba.
\end{enumerate}

Kalimat dapat dikelompokkan berdasarkan kelengkapan unsur-unsur yang menyusun kalimat. Berdasakan kelengkapan unsurnya, kalimat dapat dikelompokkan ke dalam kalimat lengkap (kalimat major) dan kalimat tak lengkap (kalimat minor). Kalimat lengkap adalah kalimat yang memiliki unsur subjek dan predikat. Kalimat tak lengkap adalah kalimat yang tidak memiliki unsur subjek dan/ atau predikat. Kalimat tak lengkap umumnya muncul di dalam wacana dimana unsur yang tidak muncul tersebut sudah diketahui atau disebutkan sebelumnya \citep{alwi}.
% section section_name (end)