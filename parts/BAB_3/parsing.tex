\section{\emph{Parsing}}
Menurut \citet{bar_feigenbaum} melalui \citet{suryawan}, \emph{parsing} adalah proses delinierisasi input linguistik, yaitu menggunakan sintak dan sumber pengetahuan lain untuk menentukan fungsi dari kata-kata yang terdapat di dalam kalimat untuk membentuk suatu struktur data seperti \emph{derivation tree} yang dapat digunakan untuk memperoleh makna kalimat. \citet{bar_feigenbaum} memandang \emph{parser} sebagai \emph{recursive pattern matcher} yang melakukan pencarian untuk memetakan string kata-kata ke dalam himpunan pola sintaksis yang bermakna.

Himpunan dari pola sintaksis yang digunakan oleh parser ditentukan oleh \emph{grammar} dari kalimat input. Secara teoritis, parser dapat memutuskan kalimat-kalimat yang termasuk dalam kalimat gramatikal dan dapat membangun struktur data yang merepresentasikan struktur sintaksis dari kalimat gramatikal yang ditemui dengan menggunakan \emph{grammar} yang kemprehensif \citep{bar_feigenbaum}.

Chomsky dan Postal dalam \citet{bar_feigenbaum} menyatakan bahwa secara umum bahasa alami tidak bersifat \emph{context-free}. Salah satu pendekatan yang digunakan untuk dapat mengakomodasi bahasa alami yang tidak bersifat \emph{context-free} adalah dengan menggunakan fitur di dalam \emph{grammar}.