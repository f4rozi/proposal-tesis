\begin{thebibliography}{99}
	\bibitem[Admojo (2015)]{admojo}
		Admojo, F.T., 2015, Sistem pencarian Informasi berbasis ontologi untuk pendakian gunung menggunakan query bahasa alami dengan penyajian peta interaktif, tesis, Program Pascasarjana S2 Ilmu Komputer, Fakultas Matematika dan Ilmu Pengetahuan Alam, Universitas Gadjah Mada, Yogyakarta.
	\bibitem[Alwi \emph{et al}(2003)]{alwi}
		Alwi, H., Dardjowidjodjo, S., Lapoliwa, H dan Moeliono, A. M., 2003, \emph{Tata Bahasa Baku Bahasa Indonesia}, Balai Pustaka, Jakarta
	\bibitem[Andri (2011)]{andri}
		Andri, 2011, Rancang Bangun Online Public Acces Catalog (OPAC) Berbasis Web Semantic, Tesis, Program Studi S2 Ilmu Komputer, Fakultas Matematika dan Ilmu Pengetahuan Alam, Universitas Gadjah Mada, Yogyakarta.
	\bibitem[Andri dan Sholichah (2006)]{azhari_sholichah}
		Azhari dan Sholichah, M., 2006, Model Ontologi untuk Informasi Jadwal Penerbangan Menggunakan Protege, Jurnal Informatika, No.1, Vol.7, Hal.J67-J76.
	\bibitem[Antoniou dan Hermelen(2008)]{antoniou}
		Antoniou, G. dan van Hermelen, F., 2008, \emph{A Semantic Web Primer}, edisi 2, The MIT Press, London
	\bibitem[Badra \emph{et al} (2011)]{badra}
		Badra, F., Paul Servant, F., Passant, A., 2011, A Semantic Web Representation Of a Product Range Specication based on Constraint Satisfaction Problem in Automotive Industry, Proceedings, of the 1st international Workshop on ontology and semantic web for manufacturing, heraklion, create, greece.
	\bibitem[Bar dan Feigenbaum(1981)]{bar_feigenbaum}
		Bar, A. dan Feigenbaum, E.A., 1981, \emph{The Handbook of Artificial Intelligence Volume I}, William Kaufman Inc, California, USA
	\bibitem[Bendi (2010)]{bendi}
		Bendi, 2010, Sistem Question Answering Sederhana Berbasis Ontologi sebagai Aplikasi Web Semantic, Tesis, Program Studi S2 Ilmu Komputer, Fakultas Matematika dan Ilmu Pengetahuan Alam, Universitas Gadjah Mada, Yogyakarta.
	\bibitem[Berners \emph{et al} (2001)]{berners}
		Berner-lee, T., Hendler, J., dan Lasilla, O., 2001, The Semantic Web. American Scientific, USA.
	\bibitem[Brietman \emph{et al}(2007)]{brietman}
		Brietman, K. K., Casanova, M. A. dan Truszkowski, W., 2007, Semantic Web: Concept, Technologies and Aplications, Springer, London.
	\bibitem[Chaer(2006)]{chaer}
		Chaer, A., 2006, \emph{Tata Bahasa Praktis Bahasa Indonesia}, Rineka Cipta, Jakarta
	\bibitem[Dardjowidjojo(2010)]{dardjo}
		Dardjowidjojo, S., 2010, \emph{Psikolinguistik: Pengantar Pemahaman Bahasa Manusia}, Yayasan Obor Indonesia, Jakarta
	\bibitem[Haryawan (2014)]{haryawan}
		Haryawan, C., 2014, Pemanfaatan Sparql inferencing notation(SPIN) dalam prototipe pencarian semantik pada data restoran,tesis, Program Studi S2 Ilmu Komputer, Fakultas Matematika dan Ilmu Pengetahuan Alam, Universitas Gadjah Mada, Yogyakarta.
	\bibitem[McGuinness dan van Harmelen(2004)]{mcguinness_vanharmelen}
		McGuinness, D. L dan Van-Harmelen, F., 2004, OWL Web Ontology Language Overview, \emph{http://www.w3.org/TR/2004/REC-owl-features-20040210/}, diakses tanggal 27 Pebruari 2016
	\bibitem[Riswanto (2012)]{riswanto}
		Riswanto, E., 2012, Rancang Bangun Model Semantic Search Dengan Metode Rule Based Sebagai Aplikasi Web Semantic (Studi Kasus Pada Informasi Musik), Tesis, Program Studi S2 Ilmu Komputer, Fakultas Matematika dan Ilmu Pengetahuan Alam, Universitas Gadjah Mada, Yogyakarta.
	\bibitem[Unschold dan Gruinner (1996)]{unschold_gruinner}
		Unschold, M dan Gruinner, M., 1996, Ontologies: Principles, Methods and Applications, \emph{Knowledge Engineering Review} 11-2
	\bibitem[Sirin dan Parsia(2007)]{evren_parsia}
		Sirin, E dan Parsia, B., 2007, SPARQL-DL: SPARQL Query for OWL-DL, \empty{OWL:Experiences and Directions}, Innsbruck, Austria.
	\bibitem[Yu(2010)]{liyang_yu}
		Yu. L., 2010, \emph{A Developer's Guide to the Semantic Web}, Springer, New York, USA
\end{thebibliography}
