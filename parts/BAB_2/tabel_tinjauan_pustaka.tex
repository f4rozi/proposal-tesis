\begin{landscape}
\begin{table}[h]
	\caption{Perbandingan penelitian dan metode yang digunakan}
	\label{table:perbandingan_penelitian}
	\begin{center}
	\begin{tabular}{|c|c|c|c|c|}
		\hline
		No & Peneliti & Model Sistem & Ontologi & Domain \\
		\hline
		1 & Bendi, 2010 & Sistem \emph{Question Answering} untuk informasi data perfilman & \emph{Sparql} & \emph{Closed} \\
		\hline
		2 & Andri, 2011 & Sistem pencarian untuk domain data perpustakaan & \emph{Sparql} & \emph{Closed} \\
		\hline
		3 & Riswanto, 2012 & Semantic Search dengan metode rule based pada informasi music & \emph{SWRL} & \emph{Closed} \\
		\hline
		4 & Badra, dkk 2011 & pemanfatan \emph{semantic web} untuk industri automotif & \emph{Sparql} dan \emph{SPIN} & \emph{Closed} \\
		\hline
		5 & Haryawan, 2014 & Pemanfaatan SPIN untuk pencaraian data restoran & \emph{sparql} dan \emph{SPIN} & \emph{Closed} \\
		\hline
		6 & Admojo, 2015 & Pengembangan sistem pencarian untuk domain informasi jalur pendakian gunung  & \emph{Sparql}  & \emph{Closed} \\
		\hline
		7 & Azhari dan Sholichah (2006) & pengembangan model ontologi untuk jadwal penerbangan & \emph{Sparql} & \emph{Closed} \\
		\hline
		8 & Fahrurrozi & Membangun \emph{Plugin} untuk Protege & \emph{Sparql-DL} & \emph{Open} \\
		\hline
	\end{tabular}
	\end{center}
\end{table}
\end{landscape}