\section{Tinjauan Pustaka}
penelitian yang berkaitan dengan rancang bangun plugin protege untuk memperoses Query OWL-DL menggunakan bahasa alami belum banyak dilakukan oleh peneliti-peneliti sebelumnya seperti terlihat dalam Tabel \ref{table:perbandingan_penelitian}.

\citet{azhari_sholichah}melakukan pengembangan untuk jadwal penerbangan pesawat dengan menggunakan pemodelan ontologi, sehingga pencarian informasi dapat dibentuk berdasarkan persepsi pengguna dan informasi dapat dideskripsikan secara lebih semantik. penelitian serupa juga dilakukan oleh \citet{andri} dengan membangun aplikasi untuk pencarian data pada domain perpustakaan dengan memanfaatkan OWL, dimana proses pencarian menggunakan input bahasa alami, aplikasi yang dibangun mampu memperoses bahasa alami berupa kalimat perintah sederhana akan tetapi aplikasi yang dibangun belum mampu memperoses kalimat yang kompleks karena pemeriksaan yang dilakukan hanya didasarkan pada pola-pola kata yang sudah ditentukan tanpa melakukan analisis struktur dalam menggunakan aturan tata bahasa indonesia.

OWL memungkin kita untuk membangun analisa terstruktur menggunakan tata bahasa indonesia dengan memanfaatkan RDFs sehingga \citet{bendi} membangun aplikasi yang berkaitan dengan informasi film yang mana sumbernya didapatkan dari internet movie database, dengan cara memodelkan data film menggunakan OWL. Proses pencarian dilakukan dengan mencari informasi data yang sudah tersimpan di dalam OWL, dimana aplikasi yang dihasilkan mampu memperoses pertanyaan-pertayaan yang bersifat faktual yang melibatkan Subjek Predikat Objek.

Di tahun selanjutnya, \citet{riswanto} membangun aplikasi pencarian berbasis semantik dengan memanfaatkan SWRL pada domain Musik, Riswanto dalam penelitiannya mengijinkan untuk menuliskan kalimat pencarian tanpa aturan penulisan, akan tetapi pencarian ini hanya dapat mencari \emph{single value}. Pencarian \emph{multi value} seperti “ pengarang lagu yogyakarta di album KLA” tidak diperkenankan.

\citet{haryawan} mengimplementasikan \emph{sparql infrencing Notation} (SPIN) dalam aplikasi pencarian semantik pada data restoran untuk yang menghasilkan \emph{search} dan  \emph{result} yang sesuai keinginan pengguna, dimana fitur SPIN yang digunakan yaitu \emph{built-in function}, \emph{User Defined Function}(UDF), \emph{Template Query},\emph{Spin Rule} dan \emph{magic Properties} dengan hasil pengujian efektifitas menunjukkan bahwa rasio \emph{recall} dan \emph{precision} 1:1 yang artinya memiliki efektifitas dan efisiensi yang sangat tinggi. Penelitian ini dikembangkan untuk mengakomodasi fleksibilitas pencarian tanpa harus mengikuti suatu aturan tertentu.

Di tahun sebelumnya juga \citet{badra} melakukan penelitian pada perusahaan otomotif renault mengenai \emph{product Range Spesification} (PRS) yang dimodelkan sebagai \emph{Constrain Satisfaction Problem} (CSR). Pada penelitian ini bertujuan untuk mempelajari bagaimana merepresentasikan PRS menggunakan \emph{semantic web} dengan memanfaatkan \emph{Constrain} pada OWL dan penerapan SPIN untuk menyelesaikan masalah PRS. 

Berbeda dengan haryawan, \citet{admojo} membangun sistem untuk mengetahui rute jalur pendakian terbaik dengan menggunakan masukan bahasa alami berupa bahasa indonesia, sistem pencarian yang ditawarkan berbasis pengetahuan semantik dengan menggunakan \emph{query sparql} dimana informasi yang ingin disajikan berupa peta interaktif.