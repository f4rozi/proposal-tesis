\subsection{Ontologi pariwisata}
Ontologi pariwisata secara spesifik memuat fakta-fakta mengenai pariwisata yang terdapat di setiap daerah. Cakupan informasi yang ingin dicapai dalam ontologi pariwisata yaitu informasi seputar nama-nama tempat wisata, informasi mengenai destinasi wisata budaya, wisata belanja, termasuk informasi mengenai akomodasi berupa penginapan yang terdapat di sekitar destinasi serta informasi mengenai transportasi untuk mencapai tempat wisata. Adapun kelas yang akan dibuat dalam ontologi pariwisata diperlihatkan dalam Tabel \ref{tab:ontopar_class}.

Ontologi pariwisata terdiri dari lima buah kelas sederhana dan enam belas buah kelas kompleks. Kelas-kelas sederhana adalah kelas yang digunakan untuk mendefinisikan konsep yang bersifat umum, sedangkan kelas kompleks adalah kelas yang digunakan untuk mendefinisikan konsep yang lebih spesifik. Konsep-konsep umum yang terdapat di dalam ontologi pariwisata adalah \emph{Akomodasi, Pantai, Gunung, Produk} dan \emph{Wisata}. Sub kelas \emph{Wisata} merupakan kelas-kelas dengan spesifikasi yang sangat spesifik, untuk itu diberikan restriksi seperti yang ditunjukkan pada baris ke 17 - 21 dalam Tabel \ref{tab:ontopar_class}. Individual kelas \emph{Wisata} dan semua sub-kelasnya tidak akan diberikan secara langsung, melainkan hanya akan didapatkan dari proses \emph{reasoning}.

\begin{longtabu}{|l|l|X|}
	\caption{Daftar kelas ontologi pariwisata}\label{tab:ontopar_class} \\ \hline
	\textbf{No} & \textbf{Nama Kelas} & \textbf{Ekspresi Restriksi} \\ \hline
	\endfirsthead
	\multicolumn{3}{c}%
	{\tablename\ \thetable\ {(lanjutan)}} \\ \hline
	\textbf{No} & \textbf{Nama Kelas} & \textbf{Ekspresi Restriksi} \\ \hline
	\endhead
	1	& 	Akomodasi	&	- \\ \hline 
	2	&	Gunung	&	- \\ \hline
	3	&	Pantai	&	- \\ \hline
	4	&	Produk	&	- \\ \hline
	5	&	Wisata	&	- \\ \hline
	6	&	Penginapan	&	\begin{math}\equiv Akomodasi \cap \lnot(Transportasi) \end{math} \\ \hline
	7	&	Hotel 	&	\begin{math}\equiv Penginapan \cap (\lnot(Losmen \cup Villa)) \end{math} \\ \hline
	8	&	Losmen	&	\begin{math}\equiv Penginapan \cap (\lnot(Hotel \cup Villa)) \end{math} \\ \hline
	9	&	Villa	&	\begin{math}\equiv Penginapan \cap (\lnot(Losmen \cup Hotel)) \end{math} \\ \hline
	10	&	Transportasi	&	\begin{math}\equiv Akomodasi \cap \lnot(Penginapan) \end{math} \\ \hline
	11	&	Tradisi	&	\begin{math}\equiv (Budaya \cap Produk) \end{math} \\ \hline
	12	&	Kuliner	&	\begin{math}\equiv Produk \cap (\lnot(Kuliner \cup Souvenir)) \end{math} \\ \hline
	13	&	Makanan	&	\begin{math}\equiv Kuliner \cap \lnot(Minuman) \end{math} \\ \hline
	14	&	Minuman	&	\begin{math}\equiv Kuliner \cap \lnot(Makanan) \end{math} \\ \hline
	15	&	Souvenir	&	\begin{math}\equiv Produk \cap (\lnot(Budaya \cap Kuliner)) \end{math} \\ \hline
	16	&	Budaya	&	\begin{math}\equiv Tradisi \end{math} \newline \begin{math}\equiv Produk \cap (\lnot(Kuliner \cup Souvenir)) \end{math} \\ \hline
	17	&	Wisata\_alam	&	\begin{math}\subseteq Wisata \end{math} \newline \begin{math}\equiv \exists hasDestination.(Pantai \cup Gunung) \subseteq \top \end{math} \\ \hline
	18	&	Wisata\_pantai	&	\begin{math} \subseteq Wisata\_alam \end{math} \newline \begin{math}\equiv \exists hasDestination.Pantai \subseteq \top  \end{math} \\ \hline
	19	&	Wisata\_belanja	&	\begin{math}\subseteq Wisata \end{math} \newline \begin{math}\equiv \exists hasProduct.(Kuliner \cup Souvenir) \subseteq \top \end{math} \\ \hline
	20	&	Wisata\_kuliner	&	\begin{math} \subseteq Wisata\_belanja \end{math} \newline \begin{math}\equiv \exists hasProduct.Kuliner \subseteq \top  \end{math} \\ \hline
	21	&	Wisata\_budaya	&	\begin{math}\subseteq Wisata \end{math} \newline \begin{math}\equiv \exists hasRitual.Budaya \subseteq \top \end{math} \\ \hline
\end{longtabu}

Selanjutnya adalah menentukan \emph{Datatype property} yang akan dibuat dalam ontologi pariwisata. Berdasarkan analisa cakupan pengetahuan yang ingin dicapai dalam ontologi pariwisata maka \emph{Datatype property} yang akan dibuat dalam ontologi pariwisata disajikan dalam Tabel \ref{tab:ontopar_dp}. 

\begin{longtabu}{|l|l|X|}
	\caption{Daftar \emph{Datatype property} ontologi pariwisata}\label{tab:ontopar_dp} \\ \hline
	\textbf{No} & \textbf{Nama Properti} & \textbf{Ekspresi Restriksi} \\ \hline
	\endfirsthead
	\multicolumn{3}{c}%
	{\tablename\ \thetable\ {(lanjutan)}} \\ \hline
	\textbf{No} & \textbf{Nama Properti} & \textbf{Ekspresi Restriksi} \\ \hline
	\endhead
	1	& 	alamat	&	- \\ \hline 
	2	&	harga\_kamar 	&	- \\ \hline
	3	&	kode\_pos	&	- \\ \hline
	4	&	nama	&	- \\ \hline
	5	&	rating	&	- \\ \hline
	6	&	tarif	&	- \\ \hline
	7	&	telepon	&	- \\ \hline
	8	&	website	&	- \\ \hline
	9	&	hasAddress	&	\begin{math} \subseteq alamat \end{math} \\ \hline
	10	&	postCode	&	\begin{math} \subseteq kode\_pos \end{math} \\ \hline
	11	&	hasName	&	\begin{math} \subseteq nama \end{math} \\ \hline
	12	&	rate	&	\begin{math} \subseteq tarif \end{math} \\ \hline
	13	&	roomRate	&	\begin{math} \subseteq harga\_kamar \end{math} \newline \begin{math} \equiv tarif\_kamar \end{math} \\ \hline
	14	&	tarif\_kamar	&	\begin{math} \subseteq harga\_kamar \end{math} \newline \begin{math} \equiv hasAddress \end{math} \\ \hline
	15	&	bintang	&	\begin{math} \subseteq rating \end{math} \newline \begin{math} \equiv hasRating \end{math} \\ \hline
	16	&	hasRating	&	\begin{math} \subseteq rating \end{math} \newline \begin{math} \equiv bintang \end{math} \\ \hline
	17	&	telp	&	\begin{math} \subseteq telepon \end{math} \newline \begin{math} \equiv hasPhone \end{math} \\ \hline
	18	&	hasPhone	&	\begin{math} \subseteq telepon \end{math} \newline \begin{math} \equiv telp \end{math} \\ \hline
	19	&	alamat\_website	&	\begin{math} \subseteq website \end{math} \newline \begin{math} \equiv hasWebsite \end{math} \\ \hline
	20	&	hasWebsite	&	\begin{math} \subseteq website \end{math} \newline \begin{math} \equiv alamat\_website \end{math} \\ \hline
\end{longtabu}

Ontologi pariwisata juga memiliki \emph{Object property} seperti yang disajikan dalam Tabel \ref{tab:ontopar_op}. Properti \emph{isLocationOf} merupakan \emph{inverse} dari \emph{hasLocation} sedangkan properti \emph{hasLocation} sendiri memiliki karakteristik ekivalen dengan properti \emph{ada\_di, berada\_di} dan \emph{letak} sehingga secara tidak langsung properti \emph{ada\_di, berada\_di} dan \emph{letak} juga merupakan \emph{inverse} dari \emph{isLocationOf}.

\begin{longtabu}{|l|l|X|}
	\caption{Daftar \emph{Object property} ontologi pariwisata}\label{tab:ontopar_op} \\ \hline
	\textbf{No} & \textbf{Nama Properti} & \textbf{Ekspresi Restriksi} \\ \hline
	\endfirsthead
	\multicolumn{3}{c}%
	{\tablename\ \thetable\ {(lanjutan)}} \\ \hline
	\textbf{No} & \textbf{Nama Properti} & \textbf{Ekspresi Restriksi} \\ \hline
	\endhead
	1	& 	destinasi	&	- \\ \hline 
	2	&	hasAttraction	&	- \\ \hline
	3	&	hasProduct	&	- \\ \hline
	5	&	terletak\_di	&	- \\ \hline
	4	&	hasRitual	&	\begin{math} \equiv hasTradition \end{math} \\ \hline
	8	&	hasTradition	&	\begin{math} \equiv hasRitual \end{math} \\ \hline
	9	&	isLocationOf	&	\begin{math} \equiv hasLocation^- \end{math} \\ \hline
	6	&	hasDestination	&	\begin{math} \subseteq destinasi \end{math} \newline \begin{math} \equiv tujuan \end{math} \\ \hline
	7	&	tujuan 	&	\begin{math} \subseteq destinasi \end{math} \newline \begin{math} \equiv hasDestination \end{math} \\ \hline
	10	&	ada\_di	&	\begin{math} \subseteq terletak\_di \end{math} \newline \begin{math} \equiv (berada\_di \cup hasLocation \cup letak) \end{math} \\ \hline
	11	&	berada\_di	&	\begin{math} \subseteq terletak\_di \end{math} \newline \begin{math} \equiv (ada\_di \cup hasLocation \cup letak) \end{math} \\ \hline
	12	&	hasLocation	&	\begin{math} \subseteq terletak\_di \end{math} \newline \begin{math} \equiv (berada\_di \cup ada\_di \cup letak) \end{math} \newline \begin{math}\equiv isLocationOf^- \end{math} \\ \hline
	13	&	letak	&	\begin{math} \subseteq terletak\_di \end{math} \newline \begin{math} \equiv (berada\_di \cup hasLocation \cup ada\_di) \end{math} \\ \hline
\end{longtabu}
