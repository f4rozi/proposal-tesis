\section{Perancangan Ontologi}
Sistem yang akan dibangun pada penelitian ini akan menggunakan tiga buah ontologi yang berbeda, masing-masing ontologi dapat diakses secara terpisah melalui protokol http. Adapun \emph{Namespace} yang akan digunakan untuk membangun ontologi yaitu:

\begin{itemize}
	\item \emph{Namespace} kelas adalah \emph{http://semanticweb.techtalk.web\-.id/ontology\#}
	\item \emph{Namespace} \emph{Datatype property} dan \emph{Object property} adalah \emph{http://semanticweb.\\techtalk.web.id/ontology/property\#}
	\item \emph{Namespace} individual adalah \emph{http://semanticweb.techtalk.web.id/resources\#}
\end{itemize}
% kelas dan individual atau \emph{instance} ketiga ontologi dalam penelitian ini yaitu \emph{http://semanticweb.techtalk.web\-.id/ontology\#}, sedangkan \emph{namespace} untuk \emph{Object property} dan \emph{Datatype property} adalah \emph{http://semanticweb.techtalk.web.id/ontology\-/property\#}.

Kelas-kelas pada setiap ontologi memiliki tipe sederhana \emph{(simple class)} dan kelas kompleks \emph{(complex class)}. Kelas sederhana adalah kelas yang tidak memiliki restriksi, sedangkan kelas kompleks adalah kelas dengan tambahan restriksi tertentu. Salah satu contoh kelas kompleks misalnya kelas \emph{Wisata\_alam} dalam ontologi pariwisata diberikan restriksi dengan aturan seperti yang diperlihatkan dalam Persamaan \ref{eq:subclass_equation} dan \ref{eq:equivalent_equation}.

\begin{equation}
	Wisata\_alam \subseteq Wisata
	\label{eq:subclass_equation}
\end{equation}

\begin{equation}
	Wisata\_alam \equiv (Gunung \cup Pantai)
	\label{eq:equivalent_equation}
\end{equation}

Persamaan \ref{eq:subclass_equation} menyatakan bahwa kelas \emph{Wisata\_alam} merupakan sub kelas dari kelas \emph{Wisata}, sedangkan Persamaan \ref{eq:equivalent_equation} menyatakan bahwa yang termasuk anggota kelas \emph{Wisata\_alam} adalah semua individu kelas \emph{Gunung} atau \emph{Pantai}, dengan kata lain \emph{Wisata\_alam} adalah obyek wisata pantai atau gunung, sehingga dengan demikian apabila terdapat sebuah pernyataan seperti yang diperlihatkan dalam Gambar \ref{fig:assertion} (baris 1 - 4), maka dapat diturunkan fakta bahwa ``Pantai Senggigi'' adalah objek wisata alam.

\begin{figure}[hb]
	\centering
	\begin{lstlisting}[language=XML, numbers=left]
<ClassAssertion>
    <Class IRI="#Pantai"/>
    <NamedIndividual IRI="#Pantai_senggigi"/>
</ClassAssertion>

<SubObjectPropertyOf>
    <ObjectProperty IRI="/property#berada_di"/>
    <ObjectProperty IRI="/property#terletak_di"/>
</SubObjectPropertyOf>

<ObjectPropertyAssertion>
    <ObjectProperty abbreviatedIRI="prop:berada_di"/>
    <NamedIndividual IRI="#Pantai_senggigi"/>
    <NamedIndividual IRI="#Senggigi"/>
</ObjectPropertyAssertion>\end{lstlisting}
	\caption{Pernyataan \emph{Senggigi} memiliki destinasi \emph{Pantai\_senggigi} dan \emph{Pantai\_senggigi} berada di \emph{Senggigi}}
	\label{fig:assertion}
\end{figure}

\emph{Datatype property} dan \emph{Object property} utama pada masing-masing ontologi menggunakan bahasa Indonesia yang kemungkinan paling sering muncul dalam proses pembentukan kalimat tanya oleh pengguna, hal ini dimaksudkan agar memudahkan proses pembentukan query SPARQL-DL, selain itu untuk setiap properti diberikan pula beberapa sub properti yang secara makna bahasa memiliki kesamaan seperti misalnya \emph{terletak\_di} secara makna memiliki kesamaan dengan \emph{berada\_di} sehingga \emph{berada\_di} dijadikan sebagai sub properti \emph{terletak\_di} sedangkan untuk semua sub properti yang secara hirarki berkedudukan sejajar diberikan relasi \emph{Equivalent property} sehingga dengan struktur seperti ini sistem akan dapat menemukan relasi implisit antar \emph{instance}. Misalnya terdapat pernyataan seperti yang ditunjukkan dalam Gambar \ref{fig:assertion} (baris 22 - 26), maka query SPARQL-DL \ref{fig:a} akan menghasilkan jawaban yang sama dengan \ref{fig:b} meskipun di dalam ontologi (Gambar \ref{fig:assertion}) tidak terdapat pernyataan eksplisit yang menyatakan bahwa \emph{Pantai\_senggigi} terletak di \emph{Senggigi}.

\begin{figure}[hb]
	\centering
	\begin{subfigure}{1\linewidth}
		\begin{lstlisting}
PREFIX : <http://semanticweb.techtalk.web.id/ontology#>
PREFIX prop:<http://semanticweb.techtalk.web.id/ontology/property#>

SELECT ?lokasi WHERE {
	PropertyValue(?lokasi,prop:terletak_di,:Senggigi)
}\end{lstlisting}
		\caption{Query dengan properti \emph{terletak\_di}}
		\label{fig:a}
	\end{subfigure}

	\begin{subfigure}{1\linewidth}
	\begin{lstlisting}	
PREFIX : <http://semanticweb.techtalk.web.id/ontology#>
PREFIX prop:<http://semanticweb.techtalk.web.id/ontology/property#>

SELECT ?lokasi WHERE {
	PropertyValue(?lokasi,prop:berada_di,:Senggigi)
}\end{lstlisting}
	\caption{Query dengan properti \emph{berada\_di}}
	\label{fig:b}
	\end{subfigure}
	\caption{Query SPARQL-DL untuk mencari individu melalui relasi \emph{terletak\_di} dan \emph{berada\_di}}
	\label{fig:sparqldl_query}
\end{figure}

\subsection{Peracangan \emph{dataset}}
\emph{Dataset} yang merupakan data-data kabupaten diletakkan dalam berkas yang terpisah dengan ontologi dengan tujuan untuk memudahkan proses pengembangan. Setiap individual yang akan dimasukkan ke dalam dataset akan dicek terlebih dahulu di laman wikipedia. 

Individual yang memiliki laman di wikipedia akan menggunakan URI yang sama dengan hasil mapping laman wikipedia ke dalam basis data DBPedia. Misalnya ``http://id.wikipedia.org/wiki/Kabupaten\_Lombok\_Timur'' dalam basis data DBPedia di transformasikan menjadi ``http://id.dbpedia.org/resource/Kabupaten\_Lombok\_Timur'' maka URI dalam \emph{dataset} menggunakan URI yang sama dengan DBPedia.

\subsection{Ontologi pariwisata}
Ontologi pariwisata secara spesifik memuat fakta-fakta mengenai pariwisata yang terdapat di setiap daerah. Cakupan informasi yang ingin dicapai dalam ontologi pariwisata yaitu informasi seputar nama-nama tempat wisata, informasi mengenai destinasi wisata budaya, wisata belanja, termasuk informasi mengenai akomodasi berupa penginapan yang terdapat di sekitar destinasi serta informasi mengenai transportasi untuk mencapai tempat wisata. Adapun kelas yang akan dibuat dalam ontologi pariwisata diperlihatkan dalam Tabel \ref{tab:ontopar_class}.

Ontologi pariwisata terdiri dari lima buah kelas sederhana dan enam belas buah kelas kompleks. Kelas-kelas sederhana adalah kelas yang digunakan untuk mendefinisikan konsep yang bersifat umum, sedangkan kelas kompleks adalah kelas yang digunakan untuk mendefinisikan konsep yang lebih spesifik. Konsep-konsep umum yang terdapat di dalam ontologi pariwisata adalah \emph{Akomodasi, Pantai, Gunung, Produk} dan \emph{Wisata}. Sub kelas \emph{Wisata} merupakan kelas-kelas dengan spesifikasi yang sangat spesifik, untuk itu diberikan restriksi seperti yang ditunjukkan pada baris ke 17 - 21 dalam Tabel \ref{tab:ontopar_class}. Individual kelas \emph{Wisata} dan semua sub-kelasnya tidak akan diberikan secara langsung, melainkan hanya akan didapatkan dari proses \emph{reasoning}.

\begin{longtabu}{|l|l|X|}
	\caption{Daftar kelas ontologi pariwisata}\label{tab:ontopar_class} \\ \hline
	\textbf{No} & \textbf{Nama Kelas} & \textbf{Ekspresi Restriksi} \\ \hline
	\endfirsthead
	\multicolumn{3}{c}%
	{\tablename\ \thetable\ {(lanjutan)}} \\ \hline
	\textbf{No} & \textbf{Nama Kelas} & \textbf{Ekspresi Restriksi} \\ \hline
	\endhead
	1	& 	Akomodasi	&	- \\ \hline 
	2	&	Gunung	&	- \\ \hline
	3	&	Pantai	&	- \\ \hline
	4	&	Produk	&	- \\ \hline
	5	&	Wisata	&	- \\ \hline
	6	&	Penginapan	&	\begin{math}\equiv Akomodasi \cap \lnot(Transportasi) \end{math} \\ \hline
	7	&	Hotel 	&	\begin{math}\equiv Penginapan \cap (\lnot(Losmen \cup Villa)) \end{math} \\ \hline
	8	&	Losmen	&	\begin{math}\equiv Penginapan \cap (\lnot(Hotel \cup Villa)) \end{math} \\ \hline
	9	&	Villa	&	\begin{math}\equiv Penginapan \cap (\lnot(Losmen \cup Hotel)) \end{math} \\ \hline
	10	&	Transportasi	&	\begin{math}\equiv Akomodasi \cap \lnot(Penginapan) \end{math} \\ \hline
	11	&	Tradisi	&	\begin{math}\equiv (Budaya \cap Produk) \end{math} \\ \hline
	12	&	Kuliner	&	\begin{math}\equiv Produk \cap (\lnot(Kuliner \cup Souvenir)) \end{math} \\ \hline
	13	&	Makanan	&	\begin{math}\equiv Kuliner \cap \lnot(Minuman) \end{math} \\ \hline
	14	&	Minuman	&	\begin{math}\equiv Kuliner \cap \lnot(Makanan) \end{math} \\ \hline
	15	&	Souvenir	&	\begin{math}\equiv Produk \cap (\lnot(Budaya \cap Kuliner)) \end{math} \\ \hline
	16	&	Budaya	&	\begin{math}\equiv Tradisi \end{math} \newline \begin{math}\equiv Produk \cap (\lnot(Kuliner \cup Souvenir)) \end{math} \\ \hline
	17	&	Wisata\_alam	&	\begin{math}\subseteq Wisata \end{math} \newline \begin{math}\equiv \exists hasDestination.(Pantai \cup Gunung) \subseteq \top \end{math} \\ \hline
	18	&	Wisata\_pantai	&	\begin{math} \subseteq Wisata\_alam \end{math} \newline \begin{math}\equiv \exists hasDestination.Pantai \subseteq \top  \end{math} \\ \hline
	19	&	Wisata\_belanja	&	\begin{math}\subseteq Wisata \end{math} \newline \begin{math}\equiv \exists hasProduct.(Kuliner \cup Souvenir) \subseteq \top \end{math} \\ \hline
	20	&	Wisata\_kuliner	&	\begin{math} \subseteq Wisata\_belanja \end{math} \newline \begin{math}\equiv \exists hasProduct.Kuliner \subseteq \top  \end{math} \\ \hline
	21	&	Wisata\_budaya	&	\begin{math}\subseteq Wisata \end{math} \newline \begin{math}\equiv \exists hasRitual.Budaya \subseteq \top \end{math} \\ \hline
\end{longtabu}

Selanjutnya adalah menentukan \emph{Datatype property} yang akan dibuat dalam ontologi pariwisata. Berdasarkan analisa cakupan pengetahuan yang ingin dicapai dalam ontologi pariwisata maka \emph{Datatype property} yang akan dibuat dalam ontologi pariwisata disajikan dalam Tabel \ref{tab:ontopar_dp}. 

\begin{longtabu}{|l|l|X|}
	\caption{Daftar \emph{Datatype property} ontologi pariwisata}\label{tab:ontopar_dp} \\ \hline
	\textbf{No} & \textbf{Nama Properti} & \textbf{Ekspresi Restriksi} \\ \hline
	\endfirsthead
	\multicolumn{3}{c}%
	{\tablename\ \thetable\ {(lanjutan)}} \\ \hline
	\textbf{No} & \textbf{Nama Properti} & \textbf{Ekspresi Restriksi} \\ \hline
	\endhead
	1	& 	alamat	&	- \\ \hline 
	2	&	harga\_kamar 	&	- \\ \hline
	3	&	kode\_pos	&	- \\ \hline
	4	&	nama	&	- \\ \hline
	5	&	rating	&	- \\ \hline
	6	&	tarif	&	- \\ \hline
	7	&	telepon	&	- \\ \hline
	8	&	website	&	- \\ \hline
	9	&	hasAddress	&	\begin{math} \subseteq alamat \end{math} \\ \hline
	10	&	postCode	&	\begin{math} \subseteq kode\_pos \end{math} \\ \hline
	11	&	hasName	&	\begin{math} \subseteq nama \end{math} \\ \hline
	12	&	rate	&	\begin{math} \subseteq tarif \end{math} \\ \hline
	13	&	roomRate	&	\begin{math} \subseteq harga\_kamar \end{math} \newline \begin{math} \equiv tarif\_kamar \end{math} \\ \hline
	14	&	tarif\_kamar	&	\begin{math} \subseteq harga\_kamar \end{math} \newline \begin{math} \equiv hasAddress \end{math} \\ \hline
	15	&	bintang	&	\begin{math} \subseteq rating \end{math} \newline \begin{math} \equiv hasRating \end{math} \\ \hline
	16	&	hasRating	&	\begin{math} \subseteq rating \end{math} \newline \begin{math} \equiv bintang \end{math} \\ \hline
	17	&	telp	&	\begin{math} \subseteq telepon \end{math} \newline \begin{math} \equiv hasPhone \end{math} \\ \hline
	18	&	hasPhone	&	\begin{math} \subseteq telepon \end{math} \newline \begin{math} \equiv telp \end{math} \\ \hline
	19	&	alamat\_website	&	\begin{math} \subseteq website \end{math} \newline \begin{math} \equiv hasWebsite \end{math} \\ \hline
	20	&	hasWebsite	&	\begin{math} \subseteq website \end{math} \newline \begin{math} \equiv alamat\_website \end{math} \\ \hline
\end{longtabu}

Ontologi pariwisata juga memiliki \emph{Object property} seperti yang disajikan dalam Tabel \ref{tab:ontopar_op}. Properti \emph{isLocationOf} merupakan \emph{inverse} dari \emph{hasLocation} sedangkan properti \emph{hasLocation} sendiri memiliki karakteristik ekivalen dengan properti \emph{ada\_di, berada\_di} dan \emph{letak} sehingga secara tidak langsung properti \emph{ada\_di, berada\_di} dan \emph{letak} juga merupakan \emph{inverse} dari \emph{isLocationOf}.

\begin{longtabu}{|l|l|X|}
	\caption{Daftar \emph{Object property} ontologi pariwisata}\label{tab:ontopar_op} \\ \hline
	\textbf{No} & \textbf{Nama Properti} & \textbf{Ekspresi Restriksi} \\ \hline
	\endfirsthead
	\multicolumn{3}{c}%
	{\tablename\ \thetable\ {(lanjutan)}} \\ \hline
	\textbf{No} & \textbf{Nama Properti} & \textbf{Ekspresi Restriksi} \\ \hline
	\endhead
	1	& 	destinasi	&	- \\ \hline 
	2	&	hasAttraction	&	- \\ \hline
	3	&	hasProduct	&	- \\ \hline
	5	&	terletak\_di	&	- \\ \hline
	4	&	hasRitual	&	\begin{math} \equiv hasTradition \end{math} \\ \hline
	8	&	hasTradition	&	\begin{math} \equiv hasRitual \end{math} \\ \hline
	9	&	isLocationOf	&	\begin{math} \equiv hasLocation^- \end{math} \\ \hline
	6	&	hasDestination	&	\begin{math} \subseteq destinasi \end{math} \newline \begin{math} \equiv tujuan \end{math} \\ \hline
	7	&	tujuan 	&	\begin{math} \subseteq destinasi \end{math} \newline \begin{math} \equiv hasDestination \end{math} \\ \hline
	10	&	ada\_di	&	\begin{math} \subseteq terletak\_di \end{math} \newline \begin{math} \equiv (berada\_di \cup hasLocation \cup letak) \end{math} \\ \hline
	11	&	berada\_di	&	\begin{math} \subseteq terletak\_di \end{math} \newline \begin{math} \equiv (ada\_di \cup hasLocation \cup letak) \end{math} \\ \hline
	12	&	hasLocation	&	\begin{math} \subseteq terletak\_di \end{math} \newline \begin{math} \equiv (berada\_di \cup ada\_di \cup letak) \end{math} \newline \begin{math}\equiv isLocationOf^- \end{math} \\ \hline
	13	&	letak	&	\begin{math} \subseteq terletak\_di \end{math} \newline \begin{math} \equiv (berada\_di \cup hasLocation \cup ada\_di) \end{math} \\ \hline
\end{longtabu}

\subsection{Ontologi geografi}
Ontologi geografi secara spesifik menyimpan fakta-fakta geografis masing-masing daerah. Cakupan informasi yang ingin dicapai dalam ontologi geografi yaitu informasi geografis mulai dari tingkat desa hingga tingkat provinsi. Adapun informasi yang akan dimuat diantaranya adalah informasi mengenai letak geografis, populasi, kepadatan penduduk hingga komoditas asli masing-masing daerah.

Kelas yang akan dibangun dalam ontologi geografi disajikan dalam Tabel \ref{tab:ontogeo_class}. Kelas \emph{Provinsi, Kabupaten, Kota, Kecamatan, Desa, Kota} dan \emph{Komoditas} bersifat \emph{disjoint} sedangkan untuk kelas \emph{Ibu\_kota} bersifat \emph{disjoint} dengan semua kelas kecuali kelas \emph{Kota}, hal ini bertujuan untuk mengakomodasi fakta bahwa terdapat nama kota yang sekaligus berfungsi sebagai ibu kota, seperti misalnya kota \emph{mataram} yang memiliki ibu kota yang sama yaitu \emph{mataram}. Selain itu, kelas \emph{Ibu\_kota\_kabupaten} dan \emph{Ibu\_kota\_provinsi} merupakan kelas spesifik yang memiliki restriksi seperti yang disajikan dalam Tabel \ref{tab:ontogeo_class}, sehingga individual dari kelas ini bersifat implisit yang dihasilkan melalui proses \emph{reasoning}. Selain bersifat \emph{disjoint}, kelas \emph{Komoditas} hanya dapat memiliki anggota atau individual yang telah ditetapkan, oleh karena itu kelas ini bertipe \emph{enum} yang hanya dapat memiliki individual \emph{Perikanan, Perkebunan, Pertambangan, Pertanian} dan \emph{Peternakan}. Kelas \emph{Ibu\_kota} juga memiliki restriksi tambahan yaitu individual kelas ini adalah semua individual yang memiliki hubungan \emph{isCapitalOf} dengan indvidual kelas \emph{Kota}. 

\begin{longtabu}{|c|l|X|}
	\caption{Daftar kelas ontologi geografi}\label{tab:ontogeo_class} \\ \hline
	\textbf{No} & \textbf{Nama Kelas} & \textbf{Ekspresi Restriksi} \\ \hline
	\endfirsthead
	\multicolumn{3}{c}%
	{\tablename\ \thetable\ {(lanjutan)}}\\ \hline
	\textbf{No} & \textbf{Nama Kelas} & \textbf{Ekspresi Restriksi} \\ \hline
	\endhead
	1	& 	Provinsi	&	\begin{math} \lnot(Kabupaten \cup Kota \cup Kecamatan \cup Desa \cup \newline Komoditas) \end{math} \\ \hline
	2	&	Kabupaten	&	\begin{math} \lnot(Provinsi \cup Kota \cup Kecamatan \cup Desa \cup \newline Komoditas) \end{math} \\ \hline
	3	&	Kota 	&	\begin{math} \lnot(Kabupaten \cup Provinsi \cup Kecamatan \cup Desa \cup \newline Komoditas) \end{math} \\ \hline
	4	&	Kecamatan	&	\begin{math} \lnot(Kabupaten \cup Kota \cup Provinsi \cup Desa \cup \newline Komoditas) \end{math} \\ \hline
	5	&	Desa	&	\begin{math} \lnot(Kabupaten \cup Kota \cup Kecamatan \cup Provinsi \cup Komoditas) \end{math} \\ \hline
	6	&	Ibu\_kota	&	\begin{math} \equiv (\exists isCapitalOf.Kota \subseteq \top) \cap (\lnot(Kabupaten \cup Kota \cup Kecamatan \cup Desa \cup Komoditas)) \end{math}\\ \hline
	7	&	Komoditas	&	\begin{math} \{Perikanan\} \cup \{Perkebunan\} \cup \newline \{Pertambangan\} \cup \{Pertanian\} \cup \{Peternakan\} \end{math} \\ \hline
	8	&	Ibu\_kota\_kabupaten	&	\begin{math} \subseteq Ibu\_kota \end{math} \newline \begin{math} \lnot Ibu\_kota\_provinsi \end{math} \newline \begin{math} \equiv \exists isCapitalOf.Kabupaten \subseteq \top \end{math} \\ \hline
	9	&	Ibu\_kota\_provinsi		&	\begin{math} \subseteq Ibu\_kota \end{math} \newline \begin{math} \lnot Ibu\_kota\_kabupaten \end{math} \newline \begin{math} \equiv \exists isCapitalOf.Provinsi \subseteq \top \end{math} \\ \hline
\end{longtabu}

Kelas \emph{Ibu\_kota\_kabupaten} dan \emph{Ibu\_kota\_provinsi} merupakan sub kelas dari \emph{Ibu\_kota} dimana kedua kelas ini bersifat lebih spesifik untuk mendefinisikan individual yang merupakan ibu kota dari provinsi dan kabupaten. Meskipun dalam aturan restriksi tidak diberikan secara langsung mengenai sifat \emph{disjoint} dengan kelas \emph{Provinsi, Kabupaten, Kecamatan, Desa} dan \emph{Komoditas}, namun karena kelas tersebut merupakan sub kelas dari \emph{Ibu\_kota} maka secara tidak langsung keduanya juga bersifat \emph{disjoint} dengan kelas-kelas tersebut.

Kata kunci yang diperkirakan akan sering digunakan dalam membentuk pertanyaan seputar geografi diantaranya adalah alamat, letak, komoditas, kepadatan penduduk, kode pos, luas wilayah, lambang kabupaten, jumlah populasi penduduk, nomer telepon dan website. Oleh karena itu, kata-kata tersebut dijadikan sebagai properti dalam ontologi geografi. Kata-kata yang sekiranya behubungan dengan individual dalam kelas dijadikan sebagai \emph{Object property} (lihat Tabel \ref{tab:ontogeo_op}), sedangkan kata-kata yang berhubungan dengan nilai satuan baik berupa string maupun angka dijadikan sebagai \emph{Datatype property} (lihat Tabel \ref{tab:ontogeo_dp}).

\begin{longtabu}{|c|l|X|}
	\caption{Daftar \emph{Object property} ontologi geografi}\label{tab:ontogeo_op} \\ \hline
	\textbf{No} & \textbf{Nama Properti} & \textbf{Ekspresi Restriksi} \\ \hline
	\endfirsthead
	\multicolumn{3}{c}%
	{\tablename\ \thetable\ {(lanjutan)}} \\ \hline
	\textbf{No} & \textbf{Nama Properti} & \textbf{Ekspresi Restriksi} \\ \hline
	\endhead
	1	&	terletak\_di	&	- \\ \hline
	2	& 	bagain\_dari	&	- \\ \hline 
	3	&	ibu\_kota\_dari	&	- \\ \hline
	4	&	komoditas\_asli	&	- \\ \hline
	5	&	hasHead	&	\begin{math} headOf^- \end{math} \\ \hline
	6	&	headOf	&	\begin{math} hasHead^- \end{math} \\ \hline
	7	&	hasCapital 	&	\begin{math} isCapitalOf^- \end{math} \\ \hline
	8	&	hasPart	&	\begin{math} isPartOf^- \end{math} \\ \hline
	9	&	isPartOf	& \begin{math} \subseteq bagian\_dari \end{math} \newline \begin{math} hasPart^- \end{math} \\ \hline
	10	&	ada\_di	&	\begin{math} \subseteq terletak\_di \end{math}\newline \begin{math} \equiv (berada\_di \cup hasLocation \cup letak) \end{math} \\ \hline
	11	&	berada\_di	&	\begin{math} \subseteq terletak\_di \end{math}\newline \begin{math} \equiv (hasLocation \cup letak \cup ada\_di) \end{math} \\ \hline
	12	&	hasLocation	&	\begin{math} \subseteq terletak\_di \end{math}\newline \begin{math} \equiv (berada\_di \cup letak \cup ada\_di) \end{math} \\ \hline
	13	&	letak	&	\begin{math} \subseteq terletak\_di \end{math}\newline \begin{math} \equiv (berada\_di \cup hasLocation \cup ada\_di) \end{math} \\ \hline
	14	&	isCapitalOf	&	\begin{math} hasCapital^- \end{math}\newline \begin{math} \subseteq ibu\_kota\_dari \end{math} \\ \hline
	15	&	hasCommodity	&	\begin{math} \subseteq komoditas\_asli \end{math} \\ \hline
\end{longtabu}

\begin{longtabu}{|c|l|X|}
	\caption{Daftar \emph{Datatype property} ontologi geografi}\label{tab:ontogeo_dp} \\ \hline
	\textbf{No} & \textbf{Nama Properti} & \textbf{Ekspresi Restriksi} \\ \hline
	\endfirsthead
	\multicolumn{3}{c}%
	{\tablename\ \thetable\ {(lanjutan)}} \\ \hline
	\textbf{No} & \textbf{Nama Properti} & \textbf{Ekspresi Restriksi} \\ \hline
	\endhead
	1	& 	alamat	&	- \\ \hline 
	2	&	kepadatan\_penduduk 	&	- \\ \hline
	3	&	kode\_pos	&	- \\ \hline
	4	&	lambang	&	- \\ \hline
	5	&	luas\_wilayah	&	- \\ \hline
	6	&	motto	&	- \\ \hline
	7	&	populasi	&	- \\ \hline
	8	&	telepon	&	- \\ \hline
	9	&	website	&	- \\ \hline
	10	&	hasAddress	&	\begin{math} \subseteq alamat \end{math} \\ \hline
	4	&	hasDensity	&	\begin{math} \subseteq kepadatan\_penduduk \end{math} \\ \hline
	6	&	postCode	&	\begin{math} \subseteq kode\_pos \end{math} \\ \hline
	8	&	simbol	&	\begin{math} \equiv hasSymbol \end{math} \newline \begin{math} \subseteq lambang \end{math} \\ \hline
	9	&	hasSymbol	&	\begin{math} \equiv simbol \end{math} \newline \begin{math} \subseteq lambang \end{math} \\ \hline
	11	&	hasArea	&	\begin{math} \subseteq luas\_wilayah \end{math} \\ \hline
	13	&	hasMotto	&	\begin{math} \subseteq motto \end{math} \\ \hline
	15	&	hasPopulation	&	\begin{math} \subseteq populasi \end{math} \\ \hline
	16	&	telp	&	\begin{math} \equiv hasPhone \end{math} \newline \begin{math} \subseteq telepon \end{math} \\ \hline
	17	&	hasPhone	&	\begin{math} \equiv telp \end{math} \newline \begin{math} \subseteq telepon \end{math} \\ \hline
	19	&	alamat\_website	&	\begin{math} \equiv hasWebsite \end{math} \newline \begin{math} \subseteq website \end{math} \\ \hline
	20	&	hasWebsite	&	\begin{math} \equiv alamat\_website \end{math} \newline \begin{math} \subseteq website \end{math} \\ \hline
\end{longtabu}

Fakta geografis menunjukkan apabila desa \emph{x} terletak di dalam kecamatan \emph{y} dan kecamatan \emph{y} terletak di kabupaten \emph{z}, maka secara tidak langsung desa \emph{x} juga merupakan bagian dari kabupaten \emph{z}, untuk mengakomodir fakta tersebut maka properti \emph{terletak\_di} dan semua sub-propertinya diberikan sifat transitif.

\subsection{Ontologi Pemerintahan}
Ontologi pemerintahan secara spesifik menyimpan fakta-fakta mengenai informasi pemerintahan pada tingkat pemerintah daerah, mulai dari tingkat desa hingga tingkat provinsi. Ontologi pemerintahan ditujukan untuk menjawab pertanyaan-pertanyaan seputar informasi pemerintahan seperti nama kepala dinas, kepala pemerintahan mulai dari desa hingga provinsi.

Kelas yang akan dibangun dalam ontologi pemerintahan disajikan dalam Tabel \ref{tab:ontogov_class}. Ontologi pemerintahan terdiri dari dua buah kelas utama yaitu \emph{Orang} dan \emph{Pemerintah\_daerah}. Kelas \emph{Orang} adalah kelas yang bersifat umum yang terdiri dari sub-kelas dengan individual berupa pegawai negeri serta kepala-kepala daerah mulai dari tingkat desa (kepala desa) hingga provinsi (gubernur), sedangkan kelas \emph{Pemerintah\_daerah} adalah kelas yang membawahi kelas yang bersifat kelembagaan di tingkat daerah mulai dari desa hingga provinsi. Kelas \emph{Pemerintah\_daerah} bersifat \emph{disjoint} dengan kelas \emph{Orang} sehingga semua sub kelas keduanya juga secara tidak langsung bersifat \emph{disjoint}.

Secara logika, jabatan gubernur, wakil gubernur, bupati, wakil bupati, wali kota, wakil wali kota dan kepala desa adalah merupakan jabatan politis sehingga tidak dimungkinkan sebuah individu yang sedang menjabat salah satu jabatan tersebut berstatus sebagai pegawai negeri, oleh karena itu kelas-kelas tersebut bersifat \emph{disjoint}. Selain itu, seorang pegawai negeri juga tidak mungkin menjabat sebagai camat atau kepala badan, kepala biro kepala dinas serta wakil camat dalam waktu yang bersamaan, sehingga kelas-kelas tersebut juga bersifat \emph{disjoint}.

\begin{longtabu}{|c|l|X|}
	\caption{Daftar kelas ontologi pemerintahan}\label{tab:ontogov_class} \\ \hline
	\textbf{No} & \textbf{Nama Kelas} & \textbf{Ekspresi Restriksi} \\ \hline
	\endfirsthead
	\multicolumn{3}{c}%
	{\tablename\ \thetable\ {(lanjutan)}} \\ \hline
	\textbf{No} & \textbf{Nama Kelas} & \textbf{Ekspresi Restriksi} \\ \hline
	\endhead
	1	& 	Orang	&	\begin{math} \lnot(Pemerintah\_daerah) \end{math} \\ \hline 
	2	&	Pemerintah\_daerah	&	\begin{math} \lnot(Orang) \end{math} \\ \hline
	3	&	Pegawai\_negeri 	&	\begin{math} \equiv Orang \cap (\lnot (Gubernur \cup Bupati \cup Wali\_kota \cup \newline Wakil\_wali\_kota \cup Wakil\_gubernur \cup \newline Wakil\_bupati \cup Kepala\_desa)) \end{math} \\ \hline
	4	&	Bupati	&	\begin{math} \equiv Orang \cap (\exists headOf.(Kabupaten)) \cap \newline (\lnot(Gubernur \cup Kepala\_desa \cup Wakil\_bupati \cup \newline Wakil\_gubernur \cup Wali\_kota \cup Wakil\_wali\_kota)) \end{math} \\ \hline
	5	&	Gubernur	&	\begin{math} \equiv Orang \cap (\exists headOf.(Provinsi)) \cap (\lnot(Bupati \cup \newline Kepala\_desa \cup Wakil\_bupati \cup Wakil\_gubernur \cup \newline Wali\_kota \cup Wakil\_wali\_kota)) \end{math} \\ \hline
	6	&	Kepala\_desa	&	\begin{math} \equiv Orang \cap (\exists headOf.(Desa)) \cap (\lnot(Gubernur \cup \newline Bupati \cup Wakil\_bupati \cup Wakil\_gubernur \cup \newline Wali\_kota \cup Wakil\_wali\_kota)) \end{math} \\ \hline
	7	&	Wakil\_bupati	&	\begin{math} \equiv Orang \cap (\exists viceOf.(Bupati)) \cap (\lnot(Gubernur \cup \newline Kepala\_desa \cup Bupati \cup Wakil\_gubernur \cup \newline Wali\_kota \cup Wakil\_wali\_kota)) \end{math} \\ \hline
	8	&	Wakil\_gubernur	&	\begin{math} \equiv Orang \cap (\exists viceOf.(Gubernur)) \cap (\lnot(Gubernur \cup \newline Kepala\_desa \cup Bupati \cup Wakil\_bupati \cup \newline Wali\_kota \cup Wakil\_wali\_kota)) \end{math} \\ \hline
	9	&	Wakil\_wali\_kota	&	\begin{math} \equiv Orang \cap (\exists viceOf.(Wali\_kota)) \cap (\lnot(Gubernur \cup \newline Kepala\_desa \cup Bupati \cup Wakil\_gubernur \cup \newline Wali\_kota \cup Wakil\_bupati)) \end{math} \\ \hline
	10	&	Wali\_kota	&	\begin{math} \equiv Orang \cap (\exists headOf.(Kota)) \cap (\lnot(Gubernur \cup \newline Kepala\_desa \cup Bupati \cup Wakil\_gubernur \cup \newline Wakil\_bupati \cup Wakil\_wali\_kota)) \end{math} \\ \hline
	11	&	Camat	&	\begin{math} \equiv Pegawai\_negeri \cap (\exists headOf.(Kecamatan)) \cap \newline (\lnot(Kepala\_badan \cup Kepala\_biro \cup Kepala\_dinas \cup \newline Wakil\_camat)) \end{math} \\ \hline
	12	&	Kelapa\_badan	&	\begin{math} \equiv Pegawai\_negeri \cap (\exists headOf.(Badan)) \cap \newline (\lnot(Camat \cup Kepala\_biro \cup Kepala\_dinas \cup \newline Wakil\_camat)) \end{math} \\ \hline
	13	&	Kepala\_biro	&	\begin{math} \equiv Pegawai\_negeri \cap (\exists headOf.(Biro)) \cap \newline (\lnot(Kepala\_badan \cup Camat \cup Kepala\_dinas \cup \newline Wakil\_camat)) \end{math} \\ \hline
	14	&	Kepala\_dinas	&	\begin{math} \equiv Pegawai\_negeri \cap (\exists headOf.(Dinas)) \cap \newline (\lnot(Kepala\_badan \cup Kepala\_biro \cup Camat \cup \newline Wakil\_camat)) \end{math} \\ \hline
	15	&	Wakil\_camat	&	\begin{math} \equiv Pegawai\_negeri \cap (\exists viceOf.(Camat)) \cap \newline (\lnot(Kepala\_badan \cup Kepala\_biro \cup Kepala\_dinas \cup \newline Camat)) \end{math} \\ \hline
	16	&	Badan	&	\begin{math} \equiv Pemerintah\_daerah \cap (\lnot(Biro \cup Dinas \cup Desa \cup Kabupaten \cup Kecamatan \cup Kota \cup Provinsi)) \end{math} \\ \hline
	17	&	Biro	&	\begin{math} \equiv Pemerintah\_daerah \cap (\lnot(Kecamatan \cup Dinas \cup Desa \cup Kabupaten \cup Kecamatan \cup Kota \cup Provinsi)) \end{math} \\ \hline
	18	&	Dinas	&	\begin{math} \equiv Pemerintah\_daerah \cap (\lnot(Biro \cup Badan \cup Desa \cup Kabupaten \cup Kecamatan \cup Kota \cup Provinsi)) \end{math} \\ \hline
	19	&	Desa	&	\begin{math} \equiv Pemerintah\_daerah \cap (\lnot(Biro \cup Dinas \cup Badan \cup Kabupaten \cup Kecamatan \cup Kota \cup Provinsi)) \end{math} \\ \hline
	20	&	Kabupaten	&	\begin{math} \equiv Pemerintah\_daerah \cap (\lnot(Biro \cup Dinas \cup Desa \cup Badan \cup Kecamatan \cup Kota \cup Provinsi)) \end{math} \\ \hline
	21	&	Kecamatan	&	\begin{math} \equiv Pemerintah\_daerah \cap (\lnot(Biro \cup Dinas \cup Desa \cup Kabupaten \cup Badan \cup Kota \cup Provinsi)) \end{math} \\ \hline
	22	&	Kota	&	\begin{math} \equiv Pemerintah\_daerah \cap (\lnot(Biro \cup Dinas \cup Desa \cup Kabupaten \cup Kecamatan \cup Badan \cup Provinsi)) \end{math} \\ \hline
	23	&	Provinsi	&	\begin{math} \equiv Pemerintah\_daerah \cap (\lnot(Biro \cup Dinas \cup Desa \cup Kabupaten \cup Kecamatan \cup Kota \cup Badan)) \end{math} \\ \hline
\end{longtabu}

Selain kriteria \emph{disjoint}, beberapa kelas tertentu juga memiliki restriksi tambahan. Individu kelas \emph{Bupati} adalah individu yang memiliki hubungan \emph{headOf} dengan sebuah individu kelas \emph{Kabupaten}, individu kelas \emph{Gubernur} adalah individu yang memiliki hubungan \emph{headOf} dengan sebuah individu kelas \emph{Provinsi}, sedangkan untuk individu kelas wakil adalah individu yang memiliki hubungan \emph{viceOf} dengan sebuah individu yang diwakilinya seperti wakil gubernur adalah individu yang memiliki hubungan \emph{viceOf} dengan sebuah individu gubernur.

Properti yang terdapat dalam ontologi pemerintahan hanya seputar properti yang mungkin dimiliki oleh orang dan pemerintahan. \emph{Object property} yang akan dibangun dalam ontologi pemerintahan disajikan dalam Tabel \ref{tab:ontogov_op}, sedangkan untuk \emph{Datatype property} yang akan dibangun disajikan dalam Tabel \ref{tab:ontogov_dp}.

\begin{longtabu}{|c|l|X|}
	\caption{Daftar \emph{Object property} ontologi pemerintahan}\label{tab:ontogov_op} \\ \hline
	\textbf{No} & \textbf{Nama Properti} & \textbf{Ekspresi Restriksi} \\ \hline
	\endfirsthead
	\multicolumn{3}{c}%
	{\tablename\ \thetable\ {(lanjutan)}} \\ \hline
	\textbf{No} & \textbf{Nama Properti} & \textbf{Ekspresi Restriksi} \\ \hline
	\endhead
	1	&	kepala 	&	- \\ \hline
	2	&	wakil\_dari	&	- \\ \hline
	3	& 	hasVice	&	\begin{math} viceOf^- \end{math} \\ \hline 
	4	&	headOf	&	\begin{math} hasHead^- \end{math} \\ \hline
	5	&	pimpinan	&	\begin{math} \equiv hasHead \end{math} \newline \begin{math} \subseteq kepala \end{math} \\ \hline
	6	&	hasHead	&	\begin{math} headOf^- \end{math} \newline \begin{math} \subseteq kepala \end{math} \newline \begin{math} \equiv pimpinan \end{math} \\ \hline
	7	&	viceOf	&	\begin{math} viceOf^- \end{math} \newline \begin{math} \subseteq wakil\_dari \end{math} \\ \hline
\end{longtabu}

\begin{longtabu}{|c|l|X|}
	\caption{Daftar \emph{Datatype property} ontologi pemerintahan}\label{tab:ontogov_dp} \\ \hline
	\textbf{No} & \textbf{Nama Properti} & \textbf{Ekspresi Restriksi} \\ \hline
	\endfirsthead
	\multicolumn{3}{c}%
	{\tablename\ \thetable\ {(lanjutan)}} \\ \hline
	\textbf{No} & \textbf{Nama Properti} & \textbf{Ekspresi Restriksi} \\ \hline
	\endhead
	1	& 	alamat	&	- \\ \hline 
	2	&	kode\_pos	&	- \\ \hline
	3	&	telepon	&	- \\ \hline
	4	&	website	&	- \\ \hline
	5	&	hasAddress	&	\begin{math} \subseteq alamat \end{math} \\ \hline
	6	&	postCode	&	\begin{math} \subseteq kode\_pos \end{math} \\ \hline
	7	&	telp	&	\begin{math} \equiv hasPhone \end{math} \newline \begin{math} \subseteq telepon \end{math} \\ \hline
	8	&	hasPhone	&	\begin{math} \equiv telp \end{math} \newline \begin{math} \subseteq telepon \end{math} \\ \hline
	9	&	alamat\_website	&	\begin{math} \equiv hasWebsite \end{math} \newline \begin{math} \subseteq website \end{math} \\ \hline
	10	&	hasWebsite	&	\begin{math} \equiv alamat\_website \end{math} \newline \begin{math} \subseteq website \end{math} \\ \hline
\end{longtabu}

