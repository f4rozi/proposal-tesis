\subsection{Ontologi geografi}
Ontologi geografi secara spesifik menyimpan fakta-fakta geografis masing-masing daerah. Cakupan informasi yang ingin dicapai dalam ontologi geografi yaitu informasi geografis mulai dari tingkat desa hingga tingkat provinsi. Adapun informasi yang akan dimuat diantaranya adalah informasi mengenai letak geografis, populasi, kepadatan penduduk hingga komoditas asli masing-masing daerah.

Kelas yang akan dibangun dalam ontologi geografi disajikan dalam Tabel \ref{tab:ontogeo_class}. Kelas \emph{Provinsi, Kabupaten, Kota, Kecamatan, Desa, Kota} dan \emph{Komoditas} bersifat \emph{disjoint} sedangkan untuk kelas \emph{Ibu\_kota} bersifat \emph{disjoint} dengan semua kelas kecuali kelas \emph{Kota}, hal ini bertujuan untuk mengakomodasi fakta bahwa terdapat nama kota yang sekaligus berfungsi sebagai ibu kota, seperti misalnya kota \emph{mataram} yang memiliki ibu kota yang sama yaitu \emph{mataram}. Selain itu, kelas \emph{Ibu\_kota\_kabupaten} dan \emph{Ibu\_kota\_provinsi} merupakan kelas spesifik yang memiliki restriksi seperti yang disajikan dalam Tabel \ref{tab:ontogeo_class}, sehingga individual dari kelas ini bersifat implisit yang dihasilkan melalui proses \emph{reasoning}. Selain bersifat \emph{disjoint}, kelas \emph{Komoditas} hanya dapat memiliki anggota atau individual yang telah ditetapkan, oleh karena itu kelas ini bertipe \emph{enum} yang hanya dapat memiliki individual \emph{Perikanan, Perkebunan, Pertambangan, Pertanian} dan \emph{Peternakan}. Kelas \emph{Ibu\_kota} juga memiliki restriksi tambahan yaitu individual kelas ini adalah semua individual yang memiliki hubungan \emph{isCapitalOf} dengan indvidual kelas \emph{Kota}. 

\begin{longtabu}{|c|l|X|}
	\caption{Daftar kelas ontologi geografi}\label{tab:ontogeo_class} \\ \hline
	\textbf{No} & \textbf{Nama Kelas} & \textbf{Ekspresi Restriksi} \\ \hline
	\endfirsthead
	\multicolumn{3}{c}%
	{\tablename\ \thetable\ {(lanjutan)}}\\ \hline
	\textbf{No} & \textbf{Nama Kelas} & \textbf{Ekspresi Restriksi} \\ \hline
	\endhead
	1	& 	Provinsi	&	\begin{math} \lnot(Kabupaten \cup Kota \cup Kecamatan \cup Desa \cup \newline Komoditas) \end{math} \\ \hline
	2	&	Kabupaten	&	\begin{math} \lnot(Provinsi \cup Kota \cup Kecamatan \cup Desa \cup \newline Komoditas) \end{math} \\ \hline
	3	&	Kota 	&	\begin{math} \lnot(Kabupaten \cup Provinsi \cup Kecamatan \cup Desa \cup \newline Komoditas) \end{math} \\ \hline
	4	&	Kecamatan	&	\begin{math} \lnot(Kabupaten \cup Kota \cup Provinsi \cup Desa \cup \newline Komoditas) \end{math} \\ \hline
	5	&	Desa	&	\begin{math} \lnot(Kabupaten \cup Kota \cup Kecamatan \cup Provinsi \cup Komoditas) \end{math} \\ \hline
	6	&	Ibu\_kota	&	\begin{math} \equiv (\exists isCapitalOf.Kota \subseteq \top) \cap (\lnot(Kabupaten \cup Kota \cup Kecamatan \cup Desa \cup Komoditas)) \end{math}\\ \hline
	7	&	Komoditas	&	\begin{math} \{Perikanan\} \cup \{Perkebunan\} \cup \newline \{Pertambangan\} \cup \{Pertanian\} \cup \{Peternakan\} \end{math} \\ \hline
	8	&	Ibu\_kota\_kabupaten	&	\begin{math} \subseteq Ibu\_kota \end{math} \newline \begin{math} \lnot Ibu\_kota\_provinsi \end{math} \newline \begin{math} \equiv \exists isCapitalOf.Kabupaten \subseteq \top \end{math} \\ \hline
	9	&	Ibu\_kota\_provinsi		&	\begin{math} \subseteq Ibu\_kota \end{math} \newline \begin{math} \lnot Ibu\_kota\_kabupaten \end{math} \newline \begin{math} \equiv \exists isCapitalOf.Provinsi \subseteq \top \end{math} \\ \hline
\end{longtabu}

Kelas \emph{Ibu\_kota\_kabupaten} dan \emph{Ibu\_kota\_provinsi} merupakan sub kelas dari \emph{Ibu\_kota} dimana kedua kelas ini bersifat lebih spesifik untuk mendefinisikan individual yang merupakan ibu kota dari provinsi dan kabupaten. Meskipun dalam aturan restriksi tidak diberikan secara langsung mengenai sifat \emph{disjoint} dengan kelas \emph{Provinsi, Kabupaten, Kecamatan, Desa} dan \emph{Komoditas}, namun karena kelas tersebut merupakan sub kelas dari \emph{Ibu\_kota} maka secara tidak langsung keduanya juga bersifat \emph{disjoint} dengan kelas-kelas tersebut.

Kata kunci yang diperkirakan akan sering digunakan dalam membentuk pertanyaan seputar geografi diantaranya adalah alamat, letak, komoditas, kepadatan penduduk, kode pos, luas wilayah, lambang kabupaten, jumlah populasi penduduk, nomer telepon dan website. Oleh karena itu, kata-kata tersebut dijadikan sebagai properti dalam ontologi geografi. Kata-kata yang sekiranya behubungan dengan individual dalam kelas dijadikan sebagai \emph{Object property} (lihat Tabel \ref{tab:ontogeo_op}), sedangkan kata-kata yang berhubungan dengan nilai satuan baik berupa string maupun angka dijadikan sebagai \emph{Datatype property} (lihat Tabel \ref{tab:ontogeo_dp}).

\begin{longtabu}{|c|l|X|}
	\caption{Daftar \emph{Object property} ontologi geografi}\label{tab:ontogeo_op} \\ \hline
	\textbf{No} & \textbf{Nama Properti} & \textbf{Ekspresi Restriksi} \\ \hline
	\endfirsthead
	\multicolumn{3}{c}%
	{\tablename\ \thetable\ {(lanjutan)}} \\ \hline
	\textbf{No} & \textbf{Nama Properti} & \textbf{Ekspresi Restriksi} \\ \hline
	\endhead
	1	&	terletak\_di	&	- \\ \hline
	2	& 	bagain\_dari	&	- \\ \hline 
	3	&	ibu\_kota\_dari	&	- \\ \hline
	4	&	komoditas\_asli	&	- \\ \hline
	5	&	hasHead	&	\begin{math} headOf^- \end{math} \\ \hline
	6	&	headOf	&	\begin{math} hasHead^- \end{math} \\ \hline
	7	&	hasCapital 	&	\begin{math} isCapitalOf^- \end{math} \\ \hline
	8	&	hasPart	&	\begin{math} isPartOf^- \end{math} \\ \hline
	9	&	isPartOf	& \begin{math} \subseteq bagian\_dari \end{math} \newline \begin{math} hasPart^- \end{math} \\ \hline
	10	&	ada\_di	&	\begin{math} \subseteq terletak\_di \end{math}\newline \begin{math} \equiv (berada\_di \cup hasLocation \cup letak) \end{math} \\ \hline
	11	&	berada\_di	&	\begin{math} \subseteq terletak\_di \end{math}\newline \begin{math} \equiv (hasLocation \cup letak \cup ada\_di) \end{math} \\ \hline
	12	&	hasLocation	&	\begin{math} \subseteq terletak\_di \end{math}\newline \begin{math} \equiv (berada\_di \cup letak \cup ada\_di) \end{math} \\ \hline
	13	&	letak	&	\begin{math} \subseteq terletak\_di \end{math}\newline \begin{math} \equiv (berada\_di \cup hasLocation \cup ada\_di) \end{math} \\ \hline
	14	&	isCapitalOf	&	\begin{math} hasCapital^- \end{math}\newline \begin{math} \subseteq ibu\_kota\_dari \end{math} \\ \hline
	15	&	hasCommodity	&	\begin{math} \subseteq komoditas\_asli \end{math} \\ \hline
\end{longtabu}

\begin{longtabu}{|c|l|X|}
	\caption{Daftar \emph{Datatype property} ontologi geografi}\label{tab:ontogeo_dp} \\ \hline
	\textbf{No} & \textbf{Nama Properti} & \textbf{Ekspresi Restriksi} \\ \hline
	\endfirsthead
	\multicolumn{3}{c}%
	{\tablename\ \thetable\ {(lanjutan)}} \\ \hline
	\textbf{No} & \textbf{Nama Properti} & \textbf{Ekspresi Restriksi} \\ \hline
	\endhead
	1	& 	alamat	&	- \\ \hline 
	2	&	kepadatan\_penduduk 	&	- \\ \hline
	3	&	kode\_pos	&	- \\ \hline
	4	&	lambang	&	- \\ \hline
	5	&	luas\_wilayah	&	- \\ \hline
	6	&	motto	&	- \\ \hline
	7	&	populasi	&	- \\ \hline
	8	&	telepon	&	- \\ \hline
	9	&	website	&	- \\ \hline
	10	&	hasAddress	&	\begin{math} \subseteq alamat \end{math} \\ \hline
	4	&	hasDensity	&	\begin{math} \subseteq kepadatan\_penduduk \end{math} \\ \hline
	6	&	postCode	&	\begin{math} \subseteq kode\_pos \end{math} \\ \hline
	8	&	simbol	&	\begin{math} \equiv hasSymbol \end{math} \newline \begin{math} \subseteq lambang \end{math} \\ \hline
	9	&	hasSymbol	&	\begin{math} \equiv simbol \end{math} \newline \begin{math} \subseteq lambang \end{math} \\ \hline
	11	&	hasArea	&	\begin{math} \subseteq luas\_wilayah \end{math} \\ \hline
	13	&	hasMotto	&	\begin{math} \subseteq motto \end{math} \\ \hline
	15	&	hasPopulation	&	\begin{math} \subseteq populasi \end{math} \\ \hline
	16	&	telp	&	\begin{math} \equiv hasPhone \end{math} \newline \begin{math} \subseteq telepon \end{math} \\ \hline
	17	&	hasPhone	&	\begin{math} \equiv telp \end{math} \newline \begin{math} \subseteq telepon \end{math} \\ \hline
	19	&	alamat\_website	&	\begin{math} \equiv hasWebsite \end{math} \newline \begin{math} \subseteq website \end{math} \\ \hline
	20	&	hasWebsite	&	\begin{math} \equiv alamat\_website \end{math} \newline \begin{math} \subseteq website \end{math} \\ \hline
\end{longtabu}

Fakta geografis menunjukkan apabila desa \emph{x} terletak di dalam kecamatan \emph{y} dan kecamatan \emph{y} terletak di kabupaten \emph{z}, maka secara tidak langsung desa \emph{x} juga merupakan bagian dari kabupaten \emph{z}, untuk mengakomodir fakta tersebut maka properti \emph{terletak\_di} dan semua sub-propertinya diberikan sifat transitif.
